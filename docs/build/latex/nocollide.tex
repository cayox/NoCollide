%% Generated by Sphinx.
\def\sphinxdocclass{report}
\documentclass[letterpaper,10pt,english]{sphinxmanual}
\ifdefined\pdfpxdimen
   \let\sphinxpxdimen\pdfpxdimen\else\newdimen\sphinxpxdimen
\fi \sphinxpxdimen=.75bp\relax

\PassOptionsToPackage{warn}{textcomp}
\usepackage[utf8]{inputenc}
\ifdefined\DeclareUnicodeCharacter
% support both utf8 and utf8x syntaxes
  \ifdefined\DeclareUnicodeCharacterAsOptional
    \def\sphinxDUC#1{\DeclareUnicodeCharacter{"#1}}
  \else
    \let\sphinxDUC\DeclareUnicodeCharacter
  \fi
  \sphinxDUC{00A0}{\nobreakspace}
  \sphinxDUC{2500}{\sphinxunichar{2500}}
  \sphinxDUC{2502}{\sphinxunichar{2502}}
  \sphinxDUC{2514}{\sphinxunichar{2514}}
  \sphinxDUC{251C}{\sphinxunichar{251C}}
  \sphinxDUC{2572}{\textbackslash}
\fi
\usepackage{cmap}
\usepackage[T1]{fontenc}
\usepackage{amsmath,amssymb,amstext}
\usepackage{babel}



\usepackage{times}
\expandafter\ifx\csname T@LGR\endcsname\relax
\else
% LGR was declared as font encoding
  \substitutefont{LGR}{\rmdefault}{cmr}
  \substitutefont{LGR}{\sfdefault}{cmss}
  \substitutefont{LGR}{\ttdefault}{cmtt}
\fi
\expandafter\ifx\csname T@X2\endcsname\relax
  \expandafter\ifx\csname T@T2A\endcsname\relax
  \else
  % T2A was declared as font encoding
    \substitutefont{T2A}{\rmdefault}{cmr}
    \substitutefont{T2A}{\sfdefault}{cmss}
    \substitutefont{T2A}{\ttdefault}{cmtt}
  \fi
\else
% X2 was declared as font encoding
  \substitutefont{X2}{\rmdefault}{cmr}
  \substitutefont{X2}{\sfdefault}{cmss}
  \substitutefont{X2}{\ttdefault}{cmtt}
\fi


\usepackage[Bjarne]{fncychap}
\usepackage{sphinx}

\fvset{fontsize=\small}
\usepackage{geometry}


% Include hyperref last.
\usepackage{hyperref}
% Fix anchor placement for figures with captions.
\usepackage{hypcap}% it must be loaded after hyperref.
% Set up styles of URL: it should be placed after hyperref.
\urlstyle{same}

\addto\captionsenglish{\renewcommand{\contentsname}{Contents:}}

\usepackage{sphinxmessages}
\setcounter{tocdepth}{1}



\title{noCollide}
\date{Feb 14, 2021}
\release{}
\author{Nico Päller, Manuel Wilke}
\newcommand{\sphinxlogo}{\vbox{}}
\renewcommand{\releasename}{}
\makeindex
\begin{document}

\pagestyle{empty}
\sphinxmaketitle
\pagestyle{plain}
\sphinxtableofcontents
\pagestyle{normal}
\phantomsection\label{\detokenize{index::doc}}


The noCollide is a driver warning system to avoid collisions with objects. It can be implemented using LiDAR Sensors and
a Raspberry Pi or the CARLA Simulator.

\index{NoCollide@\spxentry{NoCollide}}\ignorespaces 

\chapter{NoCollide}
\label{\detokenize{no_collide:nocollide}}\label{\detokenize{no_collide:index-0}}\label{\detokenize{no_collide::doc}}
The NoCollide class is the brain of the system. Here all the sensor information is collected and used to calculate the
Time\sphinxhyphen{}to\sphinxhyphen{}Collision
\index{TtcTimes (class in lib.nocollide)@\spxentry{TtcTimes}\spxextra{class in lib.nocollide}}

\begin{fulllineitems}
\phantomsection\label{\detokenize{no_collide:lib.nocollide.TtcTimes}}\pysiglinewithargsret{\sphinxbfcode{\sphinxupquote{class }}\sphinxcode{\sphinxupquote{lib.nocollide.}}\sphinxbfcode{\sphinxupquote{TtcTimes}}}{\emph{\DUrole{n}{too\_late}\DUrole{p}{:} \DUrole{n}{float}}, \emph{\DUrole{n}{brake}\DUrole{p}{:} \DUrole{n}{float}}, \emph{\DUrole{n}{warning}\DUrole{p}{:} \DUrole{n}{float}}, \emph{\DUrole{n}{reaction\_time}\DUrole{p}{:} \DUrole{n}{float} \DUrole{o}{=} \DUrole{default_value}{0.5}}}{}
A struct like class to enhance readability when storing Time\sphinxhyphen{}to\sphinxhyphen{}Collision values.
\sphinxcode{\sphinxupquote{too\_late \textless{} brake \textless{} warning}} must be True
\begin{quote}\begin{description}
\item[{Parameters}] \leavevmode\begin{itemize}
\item {} 
\sphinxstyleliteralstrong{\sphinxupquote{too\_late}} (\sphinxstyleliteralemphasis{\sphinxupquote{float}}) \textendash{} The time after which an accident is unavoidable

\item {} 
\sphinxstyleliteralstrong{\sphinxupquote{brake}} (\sphinxstyleliteralemphasis{\sphinxupquote{float}}) \textendash{} The time after which the driver must be braking to avoid an accident

\item {} 
\sphinxstyleliteralstrong{\sphinxupquote{warning}} (\sphinxstyleliteralemphasis{\sphinxupquote{float}}) \textendash{} The time after which the driver should be warned

\item {} 
\sphinxstyleliteralstrong{\sphinxupquote{reaction\_time}} (\sphinxstyleliteralemphasis{\sphinxupquote{float}}) \textendash{} the reaction time that should be added to \sphinxcode{\sphinxupquote{\textasciigrave{}brake}} and \sphinxcode{\sphinxupquote{warning}}. Defaults to 0.5

\end{itemize}

\end{description}\end{quote}

\end{fulllineitems}

\index{NoCollide (class in lib.nocollide)@\spxentry{NoCollide}\spxextra{class in lib.nocollide}}

\begin{fulllineitems}
\phantomsection\label{\detokenize{no_collide:lib.nocollide.NoCollide}}\pysiglinewithargsret{\sphinxbfcode{\sphinxupquote{class }}\sphinxcode{\sphinxupquote{lib.nocollide.}}\sphinxbfcode{\sphinxupquote{NoCollide}}}{\emph{\DUrole{n}{driver}\DUrole{p}{:} \DUrole{n}{{\hyperref[\detokenize{driver:lib.driver.Driver}]{\sphinxcrossref{lib.driver.Driver}}}}}, \emph{\DUrole{n}{sensors}\DUrole{p}{:} \DUrole{n}{{\hyperref[\detokenize{sensor:lib.sensor.SensorGroup}]{\sphinxcrossref{lib.sensor.SensorGroup}}}}}, \emph{\DUrole{n}{ttc\_times}\DUrole{p}{:} \DUrole{n}{Optional\DUrole{p}{{[}}{\hyperref[\detokenize{no_collide:lib.nocollide.TtcTimes}]{\sphinxcrossref{lib.nocollide.TtcTimes}}}\DUrole{p}{{]}}} \DUrole{o}{=} \DUrole{default_value}{None}}}{}
The class that calculates the acc and regulates the warning.
\begin{quote}\begin{description}
\item[{Parameters}] \leavevmode\begin{itemize}
\item {} 
\sphinxstyleliteralstrong{\sphinxupquote{driver}} (\sphinxcode{\sphinxupquote{Driver}}) \textendash{} The configuration of the CAN\sphinxhyphen{}Bus to initialise

\item {} 
\sphinxstyleliteralstrong{\sphinxupquote{sensors}} ({\hyperref[\detokenize{sensor:lib.sensor.SensorGroup}]{\sphinxcrossref{\sphinxcode{\sphinxupquote{SensorGroup}}}}}) \textendash{} The sensor group, on which the calculation should be done

\item {} 
\sphinxstyleliteralstrong{\sphinxupquote{ttc\_times}} (Union{[}{\hyperref[\detokenize{no_collide:lib.nocollide.TtcTimes}]{\sphinxcrossref{\sphinxcode{\sphinxupquote{TtcTimes}}}}}, None{]}) \textendash{} The times which define when to warn based on the TTC (Time\sphinxhyphen{}to\sphinxhyphen{}collision). If None, the default
TTC\sphinxhyphen{}Times will be used: \sphinxcode{\sphinxupquote{TtcTimes(too\_late=0.6, brake=1.6, warning=2.6)}}

\end{itemize}

\end{description}\end{quote}
\index{calc() (lib.nocollide.NoCollide method)@\spxentry{calc()}\spxextra{lib.nocollide.NoCollide method}}

\begin{fulllineitems}
\phantomsection\label{\detokenize{no_collide:lib.nocollide.NoCollide.calc}}\pysiglinewithargsret{\sphinxbfcode{\sphinxupquote{calc}}}{}{}
The method that calculates the Time\sphinxhyphen{}to\sphinxhyphen{}Collision. Takes the Value of the sensor, that measures the closest object
and calculates the relative velocity. The TTC results in dividing the distance by the relative velocity
:return:

\end{fulllineitems}

\index{run() (lib.nocollide.NoCollide method)@\spxentry{run()}\spxextra{lib.nocollide.NoCollide method}}

\begin{fulllineitems}
\phantomsection\label{\detokenize{no_collide:lib.nocollide.NoCollide.run}}\pysiglinewithargsret{\sphinxbfcode{\sphinxupquote{run}}}{\emph{\DUrole{n}{block}\DUrole{p}{:} \DUrole{n}{bool} \DUrole{o}{=} \DUrole{default_value}{True}}}{}
The method to start the calculation. Can be run blocking (in an endless loop) or not (1 calulcation only)
\begin{quote}\begin{description}
\item[{Parameters}] \leavevmode
\sphinxstyleliteralstrong{\sphinxupquote{block}} (\sphinxstyleliteralemphasis{\sphinxupquote{bool}}) \textendash{} wether the method should block or not

\end{description}\end{quote}

\end{fulllineitems}

\index{warn() (lib.nocollide.NoCollide method)@\spxentry{warn()}\spxextra{lib.nocollide.NoCollide method}}

\begin{fulllineitems}
\phantomsection\label{\detokenize{no_collide:lib.nocollide.NoCollide.warn}}\pysiglinewithargsret{\sphinxbfcode{\sphinxupquote{warn}}}{\emph{\DUrole{n}{ttc}\DUrole{p}{:} \DUrole{n}{float}}}{}
The method to warn the user based on the Time\sphinxhyphen{}to\sphinxhyphen{}Collision.
\begin{quote}\begin{description}
\item[{Parameters}] \leavevmode
\sphinxstyleliteralstrong{\sphinxupquote{ttc}} (\sphinxstyleliteralemphasis{\sphinxupquote{float}}) \textendash{} the Time\sphinxhyphen{}to\sphinxhyphen{}Collision

\end{description}\end{quote}

\end{fulllineitems}


\end{fulllineitems}


\index{Sensor@\spxentry{Sensor}}\ignorespaces 

\chapter{Sensor}
\label{\detokenize{sensor:sensor}}\label{\detokenize{sensor:index-0}}\label{\detokenize{sensor::doc}}
The Sensors are the eyes and ears of the system. Here is a LiDAR Sensor implemented to be used for measuring the distance
to the object in the front.
\index{SensorInterface (class in lib.sensor)@\spxentry{SensorInterface}\spxextra{class in lib.sensor}}

\begin{fulllineitems}
\phantomsection\label{\detokenize{sensor:lib.sensor.SensorInterface}}\pysigline{\sphinxbfcode{\sphinxupquote{class }}\sphinxcode{\sphinxupquote{lib.sensor.}}\sphinxbfcode{\sphinxupquote{SensorInterface}}}
An interface to implement a sensor with all the needed methods to function properly
\index{change\_addr() (lib.sensor.SensorInterface method)@\spxentry{change\_addr()}\spxextra{lib.sensor.SensorInterface method}}

\begin{fulllineitems}
\phantomsection\label{\detokenize{sensor:lib.sensor.SensorInterface.change_addr}}\pysiglinewithargsret{\sphinxbfcode{\sphinxupquote{abstract }}\sphinxbfcode{\sphinxupquote{change\_addr}}}{\emph{\DUrole{n}{new\_addr}\DUrole{p}{:} \DUrole{n}{int}}}{}
A method to change the i2c address to be used with multiple devices on one bus
\begin{quote}\begin{description}
\item[{Parameters}] \leavevmode
\sphinxstyleliteralstrong{\sphinxupquote{new\_addr}} (\sphinxstyleliteralemphasis{\sphinxupquote{int}}) \textendash{} the new address that should be set

\end{description}\end{quote}

\end{fulllineitems}

\index{close() (lib.sensor.SensorInterface method)@\spxentry{close()}\spxextra{lib.sensor.SensorInterface method}}

\begin{fulllineitems}
\phantomsection\label{\detokenize{sensor:lib.sensor.SensorInterface.close}}\pysiglinewithargsret{\sphinxbfcode{\sphinxupquote{abstract }}\sphinxbfcode{\sphinxupquote{close}}}{}{}
A method to close the sensor/bus

\end{fulllineitems}

\index{configure() (lib.sensor.SensorInterface method)@\spxentry{configure()}\spxextra{lib.sensor.SensorInterface method}}

\begin{fulllineitems}
\phantomsection\label{\detokenize{sensor:lib.sensor.SensorInterface.configure}}\pysiglinewithargsret{\sphinxbfcode{\sphinxupquote{abstract }}\sphinxbfcode{\sphinxupquote{configure}}}{\emph{\DUrole{n}{mode}\DUrole{p}{:} \DUrole{n}{int}}}{}
A method to apply any configuration to the sensor
\begin{quote}\begin{description}
\item[{Parameters}] \leavevmode
\sphinxstyleliteralstrong{\sphinxupquote{mode}} (\sphinxstyleliteralemphasis{\sphinxupquote{int}}) \textendash{} the mode that should be set

\end{description}\end{quote}

\end{fulllineitems}

\index{measure() (lib.sensor.SensorInterface method)@\spxentry{measure()}\spxextra{lib.sensor.SensorInterface method}}

\begin{fulllineitems}
\phantomsection\label{\detokenize{sensor:lib.sensor.SensorInterface.measure}}\pysiglinewithargsret{\sphinxbfcode{\sphinxupquote{abstract }}\sphinxbfcode{\sphinxupquote{measure}}}{\emph{\DUrole{n}{rec\_bias\_corr}\DUrole{p}{:} \DUrole{n}{bool} \DUrole{o}{=} \DUrole{default_value}{True}}}{}
Method to tell the sensor to measure a value
\begin{quote}\begin{description}
\item[{Parameters}] \leavevmode
\sphinxstyleliteralstrong{\sphinxupquote{rec\_bias\_corr}} (\sphinxstyleliteralemphasis{\sphinxupquote{bool}}) \textendash{} wether to measure with bias correction

\end{description}\end{quote}

\end{fulllineitems}

\index{read\_measurements() (lib.sensor.SensorInterface method)@\spxentry{read\_measurements()}\spxextra{lib.sensor.SensorInterface method}}

\begin{fulllineitems}
\phantomsection\label{\detokenize{sensor:lib.sensor.SensorInterface.read_measurements}}\pysiglinewithargsret{\sphinxbfcode{\sphinxupquote{abstract }}\sphinxbfcode{\sphinxupquote{read\_measurements}}}{}{{ $\rightarrow$ float}}
Method to retrieve the measured data from the sensor
\begin{quote}\begin{description}
\item[{Returns}] \leavevmode
the measured data

\item[{Return type}] \leavevmode
float

\end{description}\end{quote}

\end{fulllineitems}


\end{fulllineitems}

\index{Sensor (class in lib.sensor)@\spxentry{Sensor}\spxextra{class in lib.sensor}}

\begin{fulllineitems}
\phantomsection\label{\detokenize{sensor:lib.sensor.Sensor}}\pysiglinewithargsret{\sphinxbfcode{\sphinxupquote{class }}\sphinxcode{\sphinxupquote{lib.sensor.}}\sphinxbfcode{\sphinxupquote{Sensor}}}{\emph{\DUrole{n}{i2c\_bus}\DUrole{p}{:} \DUrole{n}{int} \DUrole{o}{=} \DUrole{default_value}{1}}, \emph{\DUrole{n}{max\_range}\DUrole{p}{:} \DUrole{n}{int} \DUrole{o}{=} \DUrole{default_value}{50}}}{}
The Class that handles a LiDAR Sensor
\begin{quote}\begin{description}
\item[{Param}] \leavevmode
i2c\_bus: the bus number on which the sensor is running, defaults to Bus 1

\end{description}\end{quote}
\index{change\_addr() (lib.sensor.Sensor method)@\spxentry{change\_addr()}\spxextra{lib.sensor.Sensor method}}

\begin{fulllineitems}
\phantomsection\label{\detokenize{sensor:lib.sensor.Sensor.change_addr}}\pysiglinewithargsret{\sphinxbfcode{\sphinxupquote{change\_addr}}}{\emph{\DUrole{n}{new\_addr}\DUrole{p}{:} \DUrole{n}{int}}}{}
Method to change the I2C Address of the sensor
\begin{quote}\begin{description}
\item[{Parameters}] \leavevmode
\sphinxstyleliteralstrong{\sphinxupquote{new\_addr}} (\sphinxstyleliteralemphasis{\sphinxupquote{int}}) \textendash{} the new address that should be set

\end{description}\end{quote}

\end{fulllineitems}

\index{close() (lib.sensor.Sensor method)@\spxentry{close()}\spxextra{lib.sensor.Sensor method}}

\begin{fulllineitems}
\phantomsection\label{\detokenize{sensor:lib.sensor.Sensor.close}}\pysiglinewithargsret{\sphinxbfcode{\sphinxupquote{close}}}{}{}
Method to close the bus

\end{fulllineitems}

\index{configure() (lib.sensor.Sensor method)@\spxentry{configure()}\spxextra{lib.sensor.Sensor method}}

\begin{fulllineitems}
\phantomsection\label{\detokenize{sensor:lib.sensor.Sensor.configure}}\pysiglinewithargsret{\sphinxbfcode{\sphinxupquote{configure}}}{\emph{\DUrole{n}{mode}\DUrole{p}{:} \DUrole{n}{int} \DUrole{o}{=} \DUrole{default_value}{0}}}{}
Method to initialize the sensor to different modi. Must be done before the sensor can be used

\sphinxstylestrong{configuration:}  Defaults to \sphinxstylestrong{0}.

\begin{DUlineblock}{0em}
\item[] \sphinxstylestrong{0}: Default mode, balanced performance
\item[] \sphinxstylestrong{1}: Short range, high speed. Uses \sphinxcode{\sphinxupquote{0x1d}} maximum acquisition count
\item[] \sphinxstylestrong{2}: Default range, higher speed short range. Turns on quick termination
\item[]
\begin{DUlineblock}{\DUlineblockindent}
\item[] detection for faster measurements at short range (with decreased accuracy)
\end{DUlineblock}
\item[] \sphinxstylestrong{3}: Maximum range. Uses \sphinxcode{\sphinxupquote{0xff}} maximum acquisition count
\item[] \sphinxstylestrong{4}: High sensitivity detection. Overrides default valid measurement detection
\item[]
\begin{DUlineblock}{\DUlineblockindent}
\item[] algorithm, and uses a threshold value for high sensitivity and noise
\end{DUlineblock}
\item[] \sphinxstylestrong{5}: Low sensitivity detection. Overrides default valid measurement detection
\item[]
\begin{DUlineblock}{\DUlineblockindent}
\item[] algorithm, and uses a threshold value for low sensitivity and noise
\end{DUlineblock}
\end{DUlineblock}
\begin{quote}\begin{description}
\item[{Parameters}] \leavevmode
\sphinxstyleliteralstrong{\sphinxupquote{mode}} (\sphinxstyleliteralemphasis{\sphinxupquote{int}}) \textendash{} the selected mode

\end{description}\end{quote}

\end{fulllineitems}

\index{measure() (lib.sensor.Sensor method)@\spxentry{measure()}\spxextra{lib.sensor.Sensor method}}

\begin{fulllineitems}
\phantomsection\label{\detokenize{sensor:lib.sensor.Sensor.measure}}\pysiglinewithargsret{\sphinxbfcode{\sphinxupquote{measure}}}{\emph{\DUrole{n}{rec\_bias\_corr}\DUrole{p}{:} \DUrole{n}{bool} \DUrole{o}{=} \DUrole{default_value}{True}}}{}
Method to tell the sensor to take a measurement
\begin{quote}\begin{description}
\item[{Parameters}] \leavevmode
\sphinxstyleliteralstrong{\sphinxupquote{rec\_bias\_corr}} (\sphinxstyleliteralemphasis{\sphinxupquote{bool}}) \textendash{} Wether the measurement should be taken with or without receiver bias correction; defaults to True

\end{description}\end{quote}

\end{fulllineitems}

\index{read\_measurements() (lib.sensor.Sensor method)@\spxentry{read\_measurements()}\spxextra{lib.sensor.Sensor method}}

\begin{fulllineitems}
\phantomsection\label{\detokenize{sensor:lib.sensor.Sensor.read_measurements}}\pysiglinewithargsret{\sphinxbfcode{\sphinxupquote{read\_measurements}}}{}{{ $\rightarrow$ float}}
Method to obtain the measured distance in cm
\begin{quote}\begin{description}
\item[{Returns}] \leavevmode
the measured value in cm. Returns 0 if timeouted

\item[{Return type}] \leavevmode
int

\end{description}\end{quote}

\end{fulllineitems}


\end{fulllineitems}

\index{SensorGroup (class in lib.sensor)@\spxentry{SensorGroup}\spxextra{class in lib.sensor}}

\begin{fulllineitems}
\phantomsection\label{\detokenize{sensor:lib.sensor.SensorGroup}}\pysiglinewithargsret{\sphinxbfcode{\sphinxupquote{class }}\sphinxcode{\sphinxupquote{lib.sensor.}}\sphinxbfcode{\sphinxupquote{SensorGroup}}}{\emph{\DUrole{n}{i2c\_bus}\DUrole{p}{:} \DUrole{n}{int}}, \emph{\DUrole{n}{sensors}\DUrole{p}{:} \DUrole{n}{Optional\DUrole{p}{{[}}{\hyperref[\detokenize{sensor:lib.sensor.SensorInterface}]{\sphinxcrossref{lib.sensor.SensorInterface}}}\DUrole{p}{{]}}} \DUrole{o}{=} \DUrole{default_value}{None}}, \emph{\DUrole{n}{sensor\_names}\DUrole{p}{:} \DUrole{n}{Optional\DUrole{p}{{[}}List\DUrole{p}{{[}}str\DUrole{p}{{]}}\DUrole{p}{{]}}} \DUrole{o}{=} \DUrole{default_value}{None}}, \emph{\DUrole{n}{init\_mode}\DUrole{p}{:} \DUrole{n}{int} \DUrole{o}{=} \DUrole{default_value}{0}}}{}
Class to connect to multiple sensors at once.

This class can be used in a context manager (\sphinxcode{\sphinxupquote{with}} statement).
It returns itself and then all {\hyperref[\detokenize{sensor:lib.sensor.Sensor}]{\sphinxcrossref{\sphinxcode{\sphinxupquote{Sensors}}}}} which are being set (default: 3).

\sphinxstylestrong{If not used in a} \sphinxcode{\sphinxupquote{with}} \sphinxstylestrong{\sphinxhyphen{}Statement the bus must be closed manually using the} {\hyperref[\detokenize{sensor:lib.sensor.SensorGroup.close}]{\sphinxcrossref{\sphinxcode{\sphinxupquote{close()}}}}} \sphinxstylestrong{method}

\begin{sphinxVerbatim}[commandchars=\\\{\}]
\PYG{k}{with} \PYG{n}{SensorGroup}\PYG{p}{(}\PYG{n}{i2c\PYGZus{}bus}\PYG{o}{=}\PYG{l+m+mi}{1}\PYG{p}{)} \PYG{k}{as} \PYG{p}{(}\PYG{n}{group}\PYG{p}{,} \PYG{o}{*}\PYG{n}{sensors}\PYG{p}{)}\PYG{p}{:}
    \PYG{o}{.}\PYG{o}{.}\PYG{o}{.}
\end{sphinxVerbatim}
\begin{quote}\begin{description}
\item[{Parameters}] \leavevmode\begin{itemize}
\item {} 
\sphinxstyleliteralstrong{\sphinxupquote{i2c\_bus}} (\sphinxstyleliteralemphasis{\sphinxupquote{int}}) \textendash{} the raspberry pi bus the sensors are running on

\item {} 
\sphinxstyleliteralstrong{\sphinxupquote{sensor\_names}} (\sphinxstyleliteralemphasis{\sphinxupquote{Union}}\sphinxstyleliteralemphasis{\sphinxupquote{{[}}}\sphinxstyleliteralemphasis{\sphinxupquote{List}}\sphinxstyleliteralemphasis{\sphinxupquote{{[}}}\sphinxstyleliteralemphasis{\sphinxupquote{str}}\sphinxstyleliteralemphasis{\sphinxupquote{{]}}}\sphinxstyleliteralemphasis{\sphinxupquote{, }}\sphinxstyleliteralemphasis{\sphinxupquote{None}}\sphinxstyleliteralemphasis{\sphinxupquote{{]}}}) \textendash{} a list of names, to identify the sensors. Defaults to \sphinxcode{\sphinxupquote{{[}"left", "center", "right"{]}}} if \sphinxcode{\sphinxupquote{None}}

\item {} 
\sphinxstyleliteralstrong{\sphinxupquote{init\_mode}} (\sphinxstyleliteralemphasis{\sphinxupquote{int}}) \textendash{} the mode the sensors should be initialised with. Defaults to mode 0. See {\hyperref[\detokenize{sensor:lib.sensor.Sensor.configure}]{\sphinxcrossref{\sphinxcode{\sphinxupquote{Sensor.configure()}}}}} method

\end{itemize}

\end{description}\end{quote}
\index{close() (lib.sensor.SensorGroup method)@\spxentry{close()}\spxextra{lib.sensor.SensorGroup method}}

\begin{fulllineitems}
\phantomsection\label{\detokenize{sensor:lib.sensor.SensorGroup.close}}\pysiglinewithargsret{\sphinxbfcode{\sphinxupquote{close}}}{}{}
Method do close and delete all sensors from the group. This method is also called when exiting the \sphinxcode{\sphinxupquote{with}}\sphinxhyphen{}Statement

\end{fulllineitems}

\index{set\_mode() (lib.sensor.SensorGroup method)@\spxentry{set\_mode()}\spxextra{lib.sensor.SensorGroup method}}

\begin{fulllineitems}
\phantomsection\label{\detokenize{sensor:lib.sensor.SensorGroup.set_mode}}\pysiglinewithargsret{\sphinxbfcode{\sphinxupquote{set\_mode}}}{\emph{\DUrole{n}{mode\_num}\DUrole{p}{:} \DUrole{n}{int}}, \emph{\DUrole{n}{sensors}\DUrole{p}{:} \DUrole{n}{Optional\DUrole{p}{{[}}List\DUrole{p}{{[}}str\DUrole{p}{{]}}\DUrole{p}{{]}}} \DUrole{o}{=} \DUrole{default_value}{None}}}{}
Method to set the mode for specific or all Sensors in the group
\begin{quote}\begin{description}
\item[{Parameters}] \leavevmode\begin{itemize}
\item {} 
\sphinxstyleliteralstrong{\sphinxupquote{mode\_num}} \textendash{} the mode number, referring to the mode if the {\hyperref[\detokenize{sensor:lib.sensor.Sensor.configure}]{\sphinxcrossref{\sphinxcode{\sphinxupquote{configure()}}}}} method of the {\hyperref[\detokenize{sensor:lib.sensor.Sensor}]{\sphinxcrossref{\sphinxcode{\sphinxupquote{Sensor}}}}}

\item {} 
\sphinxstyleliteralstrong{\sphinxupquote{sensors}} \textendash{} the sensornames of which the mode should be changed

\end{itemize}

\item[{Raises}] \leavevmode
\sphinxstyleliteralstrong{\sphinxupquote{TypeError}} \textendash{} TypeError when the is no such sensor in the group

\end{description}\end{quote}

\end{fulllineitems}

\index{start() (lib.sensor.SensorGroup method)@\spxentry{start()}\spxextra{lib.sensor.SensorGroup method}}

\begin{fulllineitems}
\phantomsection\label{\detokenize{sensor:lib.sensor.SensorGroup.start}}\pysiglinewithargsret{\sphinxbfcode{\sphinxupquote{start}}}{}{}
Method to start the measurements of the sensors. Launches a Thread for each sensor, where it measures continously
in a loop

\end{fulllineitems}


\end{fulllineitems}


\index{Driver@\spxentry{Driver}}\ignorespaces 

\chapter{Driver}
\label{\detokenize{driver:driver}}\label{\detokenize{driver:index-0}}\label{\detokenize{driver::doc}}
The driver is the interface between the assistant system and the vehicle itself. The Driver will be used to retrieve the
current speed and set parameters e.g. the throttle.
\index{Driver (class in lib.driver)@\spxentry{Driver}\spxextra{class in lib.driver}}

\begin{fulllineitems}
\phantomsection\label{\detokenize{driver:lib.driver.Driver}}\pysigline{\sphinxbfcode{\sphinxupquote{class }}\sphinxcode{\sphinxupquote{lib.driver.}}\sphinxbfcode{\sphinxupquote{Driver}}}
An interface to be used by the {\hyperref[\detokenize{no_collide:lib.nocollide.NoCollide}]{\sphinxcrossref{\sphinxcode{\sphinxupquote{noCollide}}}}} class get and set parameters regarding
the vehicle
\index{get\_speed() (lib.driver.Driver method)@\spxentry{get\_speed()}\spxextra{lib.driver.Driver method}}

\begin{fulllineitems}
\phantomsection\label{\detokenize{driver:lib.driver.Driver.get_speed}}\pysiglinewithargsret{\sphinxbfcode{\sphinxupquote{abstract }}\sphinxbfcode{\sphinxupquote{get\_speed}}}{}{{ $\rightarrow$ {\hyperref[\detokenize{data:lib.data.Speed}]{\sphinxcrossref{lib.data.Speed}}}}}
A method to retrieve the current speed of the car
\begin{quote}\begin{description}
\item[{Returns}] \leavevmode
the current speed of the car

\item[{Return type}] \leavevmode
{\hyperref[\detokenize{data:lib.data.Speed}]{\sphinxcrossref{Speed}}}

\end{description}\end{quote}

\end{fulllineitems}

\index{set\_brake() (lib.driver.Driver method)@\spxentry{set\_brake()}\spxextra{lib.driver.Driver method}}

\begin{fulllineitems}
\phantomsection\label{\detokenize{driver:lib.driver.Driver.set_brake}}\pysiglinewithargsret{\sphinxbfcode{\sphinxupquote{abstract }}\sphinxbfcode{\sphinxupquote{set\_brake}}}{\emph{\DUrole{n}{val}}}{}
A method to set the brake amount of the vehicle
\begin{quote}\begin{description}
\item[{Parameters}] \leavevmode
\sphinxstyleliteralstrong{\sphinxupquote{val}} (\sphinxstyleliteralemphasis{\sphinxupquote{float}}) \textendash{} the brake amount

\end{description}\end{quote}

\end{fulllineitems}

\index{set\_throttle() (lib.driver.Driver method)@\spxentry{set\_throttle()}\spxextra{lib.driver.Driver method}}

\begin{fulllineitems}
\phantomsection\label{\detokenize{driver:lib.driver.Driver.set_throttle}}\pysiglinewithargsret{\sphinxbfcode{\sphinxupquote{abstract }}\sphinxbfcode{\sphinxupquote{set\_throttle}}}{\emph{\DUrole{n}{val}\DUrole{p}{:} \DUrole{n}{float}}}{}
A method to set the throttle of the vehicle
\begin{quote}\begin{description}
\item[{Parameters}] \leavevmode
\sphinxstyleliteralstrong{\sphinxupquote{val}} (\sphinxstyleliteralemphasis{\sphinxupquote{float}}) \textendash{} the throttle amount

\end{description}\end{quote}

\end{fulllineitems}

\index{warn() (lib.driver.Driver method)@\spxentry{warn()}\spxextra{lib.driver.Driver method}}

\begin{fulllineitems}
\phantomsection\label{\detokenize{driver:lib.driver.Driver.warn}}\pysiglinewithargsret{\sphinxbfcode{\sphinxupquote{abstract }}\sphinxbfcode{\sphinxupquote{warn}}}{}{}
A method to warn the user for possible collisions

\end{fulllineitems}


\end{fulllineitems}

\index{CanConfig (class in lib.driver)@\spxentry{CanConfig}\spxextra{class in lib.driver}}

\begin{fulllineitems}
\phantomsection\label{\detokenize{driver:lib.driver.CanConfig}}\pysigline{\sphinxbfcode{\sphinxupquote{class }}\sphinxcode{\sphinxupquote{lib.driver.}}\sphinxbfcode{\sphinxupquote{CanConfig}}}
Class to better store config details of the can bus if necessary

\end{fulllineitems}

\index{CanBus (class in lib.driver)@\spxentry{CanBus}\spxextra{class in lib.driver}}

\begin{fulllineitems}
\phantomsection\label{\detokenize{driver:lib.driver.CanBus}}\pysiglinewithargsret{\sphinxbfcode{\sphinxupquote{class }}\sphinxcode{\sphinxupquote{lib.driver.}}\sphinxbfcode{\sphinxupquote{CanBus}}}{\emph{\DUrole{n}{conf}\DUrole{p}{:} \DUrole{n}{{\hyperref[\detokenize{driver:lib.driver.CanConfig}]{\sphinxcrossref{lib.driver.CanConfig}}}}}, \emph{\DUrole{n}{q\_size}\DUrole{p}{:} \DUrole{n}{int} \DUrole{o}{=} \DUrole{default_value}{1}}}{}~\index{get\_speed() (lib.driver.CanBus method)@\spxentry{get\_speed()}\spxextra{lib.driver.CanBus method}}

\begin{fulllineitems}
\phantomsection\label{\detokenize{driver:lib.driver.CanBus.get_speed}}\pysiglinewithargsret{\sphinxbfcode{\sphinxupquote{get\_speed}}}{}{{ $\rightarrow$ {\hyperref[\detokenize{data:lib.data.Speed}]{\sphinxcrossref{lib.data.Speed}}}}}
Method to retrieve the speed from the queue
\begin{quote}\begin{description}
\item[{Returns}] \leavevmode
the latest speed sent via CAN

\item[{Return type}] \leavevmode
float

\end{description}\end{quote}

\end{fulllineitems}

\index{run\_forever() (lib.driver.CanBus method)@\spxentry{run\_forever()}\spxextra{lib.driver.CanBus method}}

\begin{fulllineitems}
\phantomsection\label{\detokenize{driver:lib.driver.CanBus.run_forever}}\pysiglinewithargsret{\sphinxbfcode{\sphinxupquote{run\_forever}}}{}{}
Method to keep in touch with the CAN\sphinxhyphen{}Bus. Runs in a thread, to avoid blocking

\end{fulllineitems}

\index{set\_brake() (lib.driver.CanBus method)@\spxentry{set\_brake()}\spxextra{lib.driver.CanBus method}}

\begin{fulllineitems}
\phantomsection\label{\detokenize{driver:lib.driver.CanBus.set_brake}}\pysiglinewithargsret{\sphinxbfcode{\sphinxupquote{set\_brake}}}{\emph{\DUrole{n}{val}}}{}
A method to set the brake amount of the vehicle
\begin{quote}\begin{description}
\item[{Parameters}] \leavevmode
\sphinxstyleliteralstrong{\sphinxupquote{val}} (\sphinxstyleliteralemphasis{\sphinxupquote{float}}) \textendash{} the brake amount

\end{description}\end{quote}

\end{fulllineitems}

\index{set\_throttle() (lib.driver.CanBus method)@\spxentry{set\_throttle()}\spxextra{lib.driver.CanBus method}}

\begin{fulllineitems}
\phantomsection\label{\detokenize{driver:lib.driver.CanBus.set_throttle}}\pysiglinewithargsret{\sphinxbfcode{\sphinxupquote{set\_throttle}}}{\emph{\DUrole{n}{val}\DUrole{p}{:} \DUrole{n}{float}}}{}
A method to set the throttle of the vehicle
\begin{quote}\begin{description}
\item[{Parameters}] \leavevmode
\sphinxstyleliteralstrong{\sphinxupquote{val}} (\sphinxstyleliteralemphasis{\sphinxupquote{float}}) \textendash{} the throttle amount

\end{description}\end{quote}

\end{fulllineitems}

\index{warn() (lib.driver.CanBus method)@\spxentry{warn()}\spxextra{lib.driver.CanBus method}}

\begin{fulllineitems}
\phantomsection\label{\detokenize{driver:lib.driver.CanBus.warn}}\pysiglinewithargsret{\sphinxbfcode{\sphinxupquote{warn}}}{}{}
A method to warn the user for possible collisions

\end{fulllineitems}


\end{fulllineitems}


\index{Data@\spxentry{Data}}\ignorespaces 

\chapter{Data}
\label{\detokenize{data:data}}\label{\detokenize{data:index-0}}\label{\detokenize{data::doc}}
To better and more easily calculate distances, and speed and to always have values and the corresponding times in one
place, a custom Data class was created.
\index{Data (class in lib.data)@\spxentry{Data}\spxextra{class in lib.data}}

\begin{fulllineitems}
\phantomsection\label{\detokenize{data:lib.data.Data}}\pysiglinewithargsret{\sphinxbfcode{\sphinxupquote{class }}\sphinxcode{\sphinxupquote{lib.data.}}\sphinxbfcode{\sphinxupquote{Data}}}{\emph{\DUrole{n}{value}\DUrole{p}{:} \DUrole{n}{float}}, \emph{\DUrole{n}{time}\DUrole{p}{:} \DUrole{n}{float}}}{}
A class to store a value and its corresponding time. Calculation with \sphinxcode{\sphinxupquote{+}}, \sphinxcode{\sphinxupquote{\sphinxhyphen{}}}, \sphinxcode{\sphinxupquote{*}}, \sphinxcode{\sphinxupquote{/}} and all types of
comparisons can be applied directly to the class.
\begin{quote}\begin{description}
\item[{Parameters}] \leavevmode\begin{itemize}
\item {} 
\sphinxstyleliteralstrong{\sphinxupquote{value}} (\sphinxstyleliteralemphasis{\sphinxupquote{float}}) \textendash{} the value that should be stored

\item {} 
\sphinxstyleliteralstrong{\sphinxupquote{time}} (\sphinxstyleliteralemphasis{\sphinxupquote{float}}) \textendash{} the corresponding time to the value

\end{itemize}

\end{description}\end{quote}

\end{fulllineitems}

\index{Speed (class in lib.data)@\spxentry{Speed}\spxextra{class in lib.data}}

\begin{fulllineitems}
\phantomsection\label{\detokenize{data:lib.data.Speed}}\pysiglinewithargsret{\sphinxbfcode{\sphinxupquote{class }}\sphinxcode{\sphinxupquote{lib.data.}}\sphinxbfcode{\sphinxupquote{Speed}}}{\emph{\DUrole{n}{value}\DUrole{p}{:} \DUrole{n}{float}}, \emph{\DUrole{n}{time}\DUrole{p}{:} \DUrole{n}{float}}}{}
A class to extend the {\hyperref[\detokenize{data:lib.data.Data}]{\sphinxcrossref{\sphinxcode{\sphinxupquote{Data}}}}}. This class stores velocity values and the
acceleration can be easily calculated
\index{acceleration() (lib.data.Speed method)@\spxentry{acceleration()}\spxextra{lib.data.Speed method}}

\begin{fulllineitems}
\phantomsection\label{\detokenize{data:lib.data.Speed.acceleration}}\pysiglinewithargsret{\sphinxbfcode{\sphinxupquote{acceleration}}}{\emph{\DUrole{n}{other}\DUrole{p}{:} \DUrole{n}{{\hyperref[\detokenize{data:lib.data.Speed}]{\sphinxcrossref{lib.data.Speed}}}}}}{{ $\rightarrow$ {\hyperref[\detokenize{data:lib.data.Data}]{\sphinxcrossref{lib.data.Data}}}}}
A method to calculate the acceleration
\begin{quote}\begin{description}
\item[{Parameters}] \leavevmode
\sphinxstyleliteralstrong{\sphinxupquote{other}} \textendash{} the Speed before with which the Acceleration should be calculated

\item[{Returns}] \leavevmode
the Acceleration

\item[{Return type}] \leavevmode
{\hyperref[\detokenize{data:lib.data.Data}]{\sphinxcrossref{Data}}}

\end{description}\end{quote}

\end{fulllineitems}


\end{fulllineitems}

\index{Distance (class in lib.data)@\spxentry{Distance}\spxextra{class in lib.data}}

\begin{fulllineitems}
\phantomsection\label{\detokenize{data:lib.data.Distance}}\pysiglinewithargsret{\sphinxbfcode{\sphinxupquote{class }}\sphinxcode{\sphinxupquote{lib.data.}}\sphinxbfcode{\sphinxupquote{Distance}}}{\emph{\DUrole{n}{value}\DUrole{p}{:} \DUrole{n}{float}}, \emph{\DUrole{n}{time}\DUrole{p}{:} \DUrole{n}{float}}}{}
A class to extend the {\hyperref[\detokenize{data:lib.data.Data}]{\sphinxcrossref{\sphinxcode{\sphinxupquote{Data}}}}}. This class stores distance values and the
velocity can be easily calculated
\index{velocity() (lib.data.Distance method)@\spxentry{velocity()}\spxextra{lib.data.Distance method}}

\begin{fulllineitems}
\phantomsection\label{\detokenize{data:lib.data.Distance.velocity}}\pysiglinewithargsret{\sphinxbfcode{\sphinxupquote{velocity}}}{\emph{\DUrole{n}{other}\DUrole{p}{:} \DUrole{n}{{\hyperref[\detokenize{data:lib.data.Distance}]{\sphinxcrossref{lib.data.Distance}}}}}}{{ $\rightarrow$ {\hyperref[\detokenize{data:lib.data.Speed}]{\sphinxcrossref{lib.data.Speed}}}}}
A method to calculate the velocity. Returns Speed with value 1 if the times are the same
\begin{quote}\begin{description}
\item[{Parameters}] \leavevmode
\sphinxstyleliteralstrong{\sphinxupquote{other}} \textendash{} the Distance before with which the speed should be calculated

\item[{Returns}] \leavevmode
the speed

\item[{Return type}] \leavevmode
{\hyperref[\detokenize{data:lib.data.Speed}]{\sphinxcrossref{Speed}}}

\end{description}\end{quote}

\end{fulllineitems}


\end{fulllineitems}


\index{Simulation Integration@\spxentry{Simulation Integration}}\ignorespaces 

\chapter{Simulation Integration}
\label{\detokenize{sim_interfaces:simulation-integration}}\label{\detokenize{sim_interfaces:index-0}}\label{\detokenize{sim_interfaces::doc}}
To establish a neatless simulation, the code that is intended to use hardware must be extended/changed to use the simulation’s
sensors. That’s why simulation classes are needed, to adapt to the challanges of merging different usecases in one API.
\index{SimNoCollide (class in lib.sim\_interfaces)@\spxentry{SimNoCollide}\spxextra{class in lib.sim\_interfaces}}

\begin{fulllineitems}
\phantomsection\label{\detokenize{sim_interfaces:lib.sim_interfaces.SimNoCollide}}\pysiglinewithargsret{\sphinxbfcode{\sphinxupquote{class }}\sphinxcode{\sphinxupquote{lib.sim\_interfaces.}}\sphinxbfcode{\sphinxupquote{SimNoCollide}}}{\emph{\DUrole{n}{hud}}, \emph{\DUrole{o}{*}\DUrole{n}{args}}, \emph{\DUrole{o}{**}\DUrole{n}{kwargs}}}{}
A class to use the NoCollide in the Carla\sphinxhyphen{}Simulator. Simply integrates the HUD of the simulator to warn via the
HUD rather than to warn via some kind of Bus
\begin{quote}\begin{description}
\item[{Parameters}] \leavevmode\begin{itemize}
\item {} 
\sphinxstyleliteralstrong{\sphinxupquote{hud}} \textendash{} The HUD of the Simulation

\item {} 
\sphinxstyleliteralstrong{\sphinxupquote{args}} \textendash{} Arguments to be passed to the parent class

\item {} 
\sphinxstyleliteralstrong{\sphinxupquote{kwargs}} \textendash{} Keyword arguments to be passed to the parent class

\end{itemize}

\end{description}\end{quote}
\index{warn() (lib.sim\_interfaces.SimNoCollide method)@\spxentry{warn()}\spxextra{lib.sim\_interfaces.SimNoCollide method}}

\begin{fulllineitems}
\phantomsection\label{\detokenize{sim_interfaces:lib.sim_interfaces.SimNoCollide.warn}}\pysiglinewithargsret{\sphinxbfcode{\sphinxupquote{warn}}}{\emph{\DUrole{n}{ttc}\DUrole{p}{:} \DUrole{n}{float}}}{}
Overwrites the parent warn Method to warn via the HUD rather than some kind of Bus
:param ttc: the Time\sphinxhyphen{}to\sphinxhyphen{}Collision in seconds
:type ttc: float

\end{fulllineitems}


\end{fulllineitems}

\index{SimSensor (class in lib.sim\_interfaces)@\spxentry{SimSensor}\spxextra{class in lib.sim\_interfaces}}

\begin{fulllineitems}
\phantomsection\label{\detokenize{sim_interfaces:lib.sim_interfaces.SimSensor}}\pysiglinewithargsret{\sphinxbfcode{\sphinxupquote{class }}\sphinxcode{\sphinxupquote{lib.sim\_interfaces.}}\sphinxbfcode{\sphinxupquote{SimSensor}}}{\emph{\DUrole{n}{sensor}}, \emph{\DUrole{n}{max\_range}\DUrole{o}{=}\DUrole{default_value}{40}}}{}
A class to implement the Carla\sphinxhyphen{}Simulator object detection sensor. Due to the constant pushing nature of the Carla sensor,
the last retrieved value and time must be saved to be ready when the NoCollide Algorithm requests the Value.
\begin{quote}\begin{description}
\item[{Parameters}] \leavevmode\begin{itemize}
\item {} 
\sphinxstyleliteralstrong{\sphinxupquote{sensor}} \textendash{} the Carla sensor to use

\item {} 
\sphinxstyleliteralstrong{\sphinxupquote{max\_range}} \textendash{} the maximum range of the Carla Sensor

\end{itemize}

\end{description}\end{quote}
\index{callback() (lib.sim\_interfaces.SimSensor method)@\spxentry{callback()}\spxextra{lib.sim\_interfaces.SimSensor method}}

\begin{fulllineitems}
\phantomsection\label{\detokenize{sim_interfaces:lib.sim_interfaces.SimSensor.callback}}\pysiglinewithargsret{\sphinxbfcode{\sphinxupquote{callback}}}{\emph{\DUrole{n}{data}}}{}
The callback method that is called, whenever the Carla sensor has measured new data. Simply stores the data,
so taht i can be requested any time
\begin{quote}\begin{description}
\item[{Parameters}] \leavevmode
\sphinxstyleliteralstrong{\sphinxupquote{data}} \textendash{} the data measured by the carla sensor

\end{description}\end{quote}

\end{fulllineitems}

\index{change\_addr() (lib.sim\_interfaces.SimSensor method)@\spxentry{change\_addr()}\spxextra{lib.sim\_interfaces.SimSensor method}}

\begin{fulllineitems}
\phantomsection\label{\detokenize{sim_interfaces:lib.sim_interfaces.SimSensor.change_addr}}\pysiglinewithargsret{\sphinxbfcode{\sphinxupquote{change\_addr}}}{\emph{\DUrole{n}{addr}\DUrole{p}{:} \DUrole{n}{int}}}{}
A method to change the i2c address to be used with multiple devices on one bus
\begin{quote}\begin{description}
\item[{Parameters}] \leavevmode
\sphinxstyleliteralstrong{\sphinxupquote{new\_addr}} (\sphinxstyleliteralemphasis{\sphinxupquote{int}}) \textendash{} the new address that should be set

\end{description}\end{quote}

\end{fulllineitems}

\index{close() (lib.sim\_interfaces.SimSensor method)@\spxentry{close()}\spxextra{lib.sim\_interfaces.SimSensor method}}

\begin{fulllineitems}
\phantomsection\label{\detokenize{sim_interfaces:lib.sim_interfaces.SimSensor.close}}\pysiglinewithargsret{\sphinxbfcode{\sphinxupquote{close}}}{}{}
A method to close the sensor/bus

\end{fulllineitems}

\index{configure() (lib.sim\_interfaces.SimSensor method)@\spxentry{configure()}\spxextra{lib.sim\_interfaces.SimSensor method}}

\begin{fulllineitems}
\phantomsection\label{\detokenize{sim_interfaces:lib.sim_interfaces.SimSensor.configure}}\pysiglinewithargsret{\sphinxbfcode{\sphinxupquote{configure}}}{\emph{\DUrole{n}{mode}\DUrole{p}{:} \DUrole{n}{int}}}{}
A method to apply any configuration to the sensor
\begin{quote}\begin{description}
\item[{Parameters}] \leavevmode
\sphinxstyleliteralstrong{\sphinxupquote{mode}} (\sphinxstyleliteralemphasis{\sphinxupquote{int}}) \textendash{} the mode that should be set

\end{description}\end{quote}

\end{fulllineitems}

\index{destroy() (lib.sim\_interfaces.SimSensor method)@\spxentry{destroy()}\spxextra{lib.sim\_interfaces.SimSensor method}}

\begin{fulllineitems}
\phantomsection\label{\detokenize{sim_interfaces:lib.sim_interfaces.SimSensor.destroy}}\pysiglinewithargsret{\sphinxbfcode{\sphinxupquote{destroy}}}{}{}
Method to destroy the Sensor to free up memory when the simulation has ended

\end{fulllineitems}

\index{listen() (lib.sim\_interfaces.SimSensor method)@\spxentry{listen()}\spxextra{lib.sim\_interfaces.SimSensor method}}

\begin{fulllineitems}
\phantomsection\label{\detokenize{sim_interfaces:lib.sim_interfaces.SimSensor.listen}}\pysiglinewithargsret{\sphinxbfcode{\sphinxupquote{listen}}}{}{}
A Method to tell the Carla sensor to use the {\hyperref[\detokenize{sim_interfaces:lib.sim_interfaces.SimSensor.callback}]{\sphinxcrossref{\sphinxcode{\sphinxupquote{callback()}}}}} method whenever a new
value is measured

\end{fulllineitems}

\index{measure() (lib.sim\_interfaces.SimSensor method)@\spxentry{measure()}\spxextra{lib.sim\_interfaces.SimSensor method}}

\begin{fulllineitems}
\phantomsection\label{\detokenize{sim_interfaces:lib.sim_interfaces.SimSensor.measure}}\pysiglinewithargsret{\sphinxbfcode{\sphinxupquote{measure}}}{\emph{\DUrole{n}{rec\_bias\_corr}\DUrole{p}{:} \DUrole{n}{bool} \DUrole{o}{=} \DUrole{default_value}{True}}}{}
Method to tell the sensor to measure a value
\begin{quote}\begin{description}
\item[{Parameters}] \leavevmode
\sphinxstyleliteralstrong{\sphinxupquote{rec\_bias\_corr}} (\sphinxstyleliteralemphasis{\sphinxupquote{bool}}) \textendash{} wether to measure with bias correction

\end{description}\end{quote}

\end{fulllineitems}

\index{read\_measurements() (lib.sim\_interfaces.SimSensor method)@\spxentry{read\_measurements()}\spxextra{lib.sim\_interfaces.SimSensor method}}

\begin{fulllineitems}
\phantomsection\label{\detokenize{sim_interfaces:lib.sim_interfaces.SimSensor.read_measurements}}\pysiglinewithargsret{\sphinxbfcode{\sphinxupquote{read\_measurements}}}{}{{ $\rightarrow$ {\hyperref[\detokenize{data:lib.data.Distance}]{\sphinxcrossref{lib.data.Distance}}}}}
The method ot retrieve the newest Data of the Sensor. If the Carla Sensor hasn’t measured anything yet,
a {\hyperref[\detokenize{data:lib.data.Distance}]{\sphinxcrossref{\sphinxcode{\sphinxupquote{Distance}}}}} object will be returned with the maximum range.
:return: the newest Distance measured by the Carla sensor
:rtype: {\hyperref[\detokenize{data:lib.data.Distance}]{\sphinxcrossref{\sphinxcode{\sphinxupquote{Distance}}}}}

\end{fulllineitems}

\index{stop() (lib.sim\_interfaces.SimSensor method)@\spxentry{stop()}\spxextra{lib.sim\_interfaces.SimSensor method}}

\begin{fulllineitems}
\phantomsection\label{\detokenize{sim_interfaces:lib.sim_interfaces.SimSensor.stop}}\pysiglinewithargsret{\sphinxbfcode{\sphinxupquote{stop}}}{}{}
A method that is needed when stopping the simulation.

\end{fulllineitems}


\end{fulllineitems}

\index{SimSensorGroup (class in lib.sim\_interfaces)@\spxentry{SimSensorGroup}\spxextra{class in lib.sim\_interfaces}}

\begin{fulllineitems}
\phantomsection\label{\detokenize{sim_interfaces:lib.sim_interfaces.SimSensorGroup}}\pysiglinewithargsret{\sphinxbfcode{\sphinxupquote{class }}\sphinxcode{\sphinxupquote{lib.sim\_interfaces.}}\sphinxbfcode{\sphinxupquote{SimSensorGroup}}}{\emph{\DUrole{n}{i2c\_bus}\DUrole{p}{:} \DUrole{n}{int}}, \emph{\DUrole{n}{sensors}\DUrole{p}{:} \DUrole{n}{Optional\DUrole{p}{{[}}{\hyperref[\detokenize{sensor:lib.sensor.SensorInterface}]{\sphinxcrossref{lib.sensor.SensorInterface}}}\DUrole{p}{{]}}} \DUrole{o}{=} \DUrole{default_value}{None}}, \emph{\DUrole{n}{sensor\_names}\DUrole{p}{:} \DUrole{n}{Optional\DUrole{p}{{[}}List\DUrole{p}{{[}}str\DUrole{p}{{]}}\DUrole{p}{{]}}} \DUrole{o}{=} \DUrole{default_value}{None}}, \emph{\DUrole{n}{init\_mode}\DUrole{p}{:} \DUrole{n}{int} \DUrole{o}{=} \DUrole{default_value}{0}}}{}
Class to overwrite the {\hyperref[\detokenize{sensor:lib.sensor.SensorGroup}]{\sphinxcrossref{\sphinxcode{\sphinxupquote{SensorGroup}}}}} class to use the simulated sensor when retrieving the distance
\index{get\_closest() (lib.sim\_interfaces.SimSensorGroup method)@\spxentry{get\_closest()}\spxextra{lib.sim\_interfaces.SimSensorGroup method}}

\begin{fulllineitems}
\phantomsection\label{\detokenize{sim_interfaces:lib.sim_interfaces.SimSensorGroup.get_closest}}\pysiglinewithargsret{\sphinxbfcode{\sphinxupquote{get\_closest}}}{}{{ $\rightarrow$ {\hyperref[\detokenize{data:lib.data.Distance}]{\sphinxcrossref{lib.data.Distance}}}}}
Class to overwrite the \sphinxcode{\sphinxupquote{get\_closest()}} method to use the simulated sensor when retrieving the distance
:return:

\end{fulllineitems}


\end{fulllineitems}



\chapter{Indices and tables}
\label{\detokenize{index:indices-and-tables}}\begin{itemize}
\item {} 
\DUrole{xref,std,std-ref}{genindex}

\item {} 
\DUrole{xref,std,std-ref}{modindex}

\item {} 
\DUrole{xref,std,std-ref}{search}

\end{itemize}



\renewcommand{\indexname}{Index}
\printindex
\end{document}